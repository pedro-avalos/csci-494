\documentclass[onecolumn, 12pt]{article}

\usepackage[T1]{fontenc}
\usepackage[utf8]{inputenc}
\usepackage{lmodern}

\usepackage[letterpaper, margin=1in]{geometry}
\usepackage{setspace}
\usepackage{titling}
\usepackage{authblk}

\usepackage[english]{babel}
\usepackage{csquotes}
\usepackage{epigraph}
\usepackage{footnote}
\usepackage[bottom, multiple]{footmisc}

\usepackage[style=ieee, backend=biber]{biblatex}
\addbibresource{sources.bib}
\SetCiteCommand{\cite}

\usepackage[hidelinks]{hyperref}

\makeatletter
\title{Sailing the High Seas: A Christian Exploration of Piracy}
\author{Pedro Avalos Jim\'enez}
\author{Niklas Anderson}
\affil{Wheaton College}
\date{\today}

\hypersetup{
  pdftitle={\@title},
  pdfauthor={Pedro Avalos Jim\'enez and Niklas Anderson},
  pdfsubject={piracy},
  pdfkeywords={%
    Christian, piracy, benefits, drawbacks, exploration, nuance, governance,
    privacy, malware, equal access, criminality, illegal content, economy%
  },
}
\makeatother

\begin{document}
\begin{refsection}
\maketitle

\epigraph{%
  \textins{T}hat ideas should freely spread from one to another over the globe,
  for the moral and mutual instruction of man, and improvement of his
  condition, seems to have been peculiarly and benevolently designed by nature,
  when she made them, like fire, expansible over all space, without lessening
  their density at any point, and like the air in which we breathe, move, and
  have our physical being, incapable of confinement or exclusive appropriation.
  Inventions then cannot, in nature, be a subject of property.%
}{Thomas Jefferson~\cite{barlow:wine}}

\begin{abstract}
  This study is a comprehensive study of piracy, particularly in the digital
  realm, from a Christian perspective. The paper delves into the ethical, legal,
  and moral implications of piracy, acknowledging both its potential benefits
  and drawbacks. The authors argue that piracy can promote equal access to
  information, challenge oppressive regimes, and protect privacy. However, they
  also highlight the risks associated with piracy, such as the distribution of
  dangerous or illegal content and its negative economic impact. The paper
  concludes with a call for a nuanced Christian response to piracy, emphasizing
  the importance of respecting authority and considering the potential
  consequences of piracy while considering the staying power and benefits of
  piracy-promoting tools and practices. This work encourages a thoughtful
  exploration of piracy, rather than a hasty dismissal or acceptance.
\end{abstract}

\section{Introduction}
\subsection{Background Information}
It is generally accepted that piracy is an illegal activity, but what is often overlooked
is the legitimate role that piracy plays in a variety of contexts. When it comes to
subverting oppressive regimes, for instance, the tools that piracy provides are
indispensable and for protecting the interests of consumers from predatory business
practices, the abilities of piracy are unmatched.

Needless to say, with every unregulated system comes abuse and piracy is no exception.
On top of that, piracy, on an individual level, is extremely difficult to combat because
of the immaterial nature of the items that would need to be protected. Ideas flow freely
between people especially when the Internet is involved, so stopping this flow becomes an
exponentially more difficult problem the more people have the restricted data. The ease
with which ideas propagate lead people like Thomas Jefferson to claim that they are
\enquote{incapable of confinement or exclusive appropriation.} Consequently, two parties form:
those that support piracy for its free and open spread of all information and those that
see piracy as a threat to the further advancement of technology and ideas.

Before proceeding further into the issue of piracy, it is important to have a grasp of
what piracy means insofar as how this paper interprets it. The dictionary defines piracy
as \enquote{the unauthorized use of another's production, invention, or conception especially in
infringement of a copyright.} While this covers the technical definition of piracy, this
paper elects to utilize a more wholistic definition. Piracy not only includes the
unauthorized use of someone's work, but it includes the tools and techniques that are
commonly used to accomplish this task such as torrent trackers, VPNs, and torrent clients.

As a Christian, how should one approach such a tool and how can it be utilized for the
betterment of God's kingdom? In the paper that follows, piracy will be analyzed through a
number of lenses, culminating in a Christian response to this issue. In doing so, the hope
is that the reader attains a fresh, wholistic understanding of the potential value of
piracy.

\subsection{Thesis/Argument}
Piracy should not be hastily thrown aside nor should it be accepted
wholeheartedly in every situation. Piracy is an excellent tool for
disseminating information, services, content, \&c. that may not have been
available any other way. As a Christian, a proper exploration of piracy should
consider the potential benefits (or circumstances that warrant piracy) and the
potential downsides (or circumstances that need special attention from
Christian developers, writers, and netizens).

\section{Arguments For Piracy}
As mentioned above, even though piracy is illegal, there is still merit to the tools of
piracy and the values behind them. The merits of piracy can be divided into three realms:
equal access to information, governance, and privacy.

\subsection{Equal Access}
At its core, piracy is primarily concerned with accessing information of all kinds,
regardless of whether someone is authorized to access that information or not. This is
what piracy is best at and, contrary to popular belief, it is not always a bad thing.

Take the restricted access to academic journal articles, for instance. Many academic
journals require all readers to pay not an insignificant sum to access research articles
that could contain insight that is valuable to the human race as a whole. Because of the
high cost of entry, certain groups are automatically excluded from reading and building
upon the research that has already been done. \citeauthor{barlow:wine} puts the situation
eloquently in \citetitle{barlow:wine} when he says, \textcquote{barlow:wine}[.]{%
  I am not comfortable with a model which will restrict inquiry to the wealthy%
}
In the case of academic journal articles, why shouldn't anyone be able to better
themselves by consuming quality research articles? All humans should be allowed
to continue to learn more about God's creation without being bound by their
financial status. Just as how someone can go to the library and checkout any
book that tickles their fancy, they should be freely able to learn from the
ongoing research of our world. Of course one should recognize that not all
information is safe or helpful to be freely available on the internet, but that
is a topic for a later section.

A more concrete instance of where paywalls can be directly harmful to the human race is
with regards to medical research. \citeauthor{till:medical-literature} stresses how important
access to medical research is in their article \citetitle{till:medical-literature}.
\blockcquote{till:medical-literature}[.]{%
  Access to the medical literature is essential for both the practice of
  evidence-based medicine and meaningful contribution to medical sciences.
  Nonetheless, only 12\% of newly published papers are freely accessible online,
  and, as of 2014, only 3 million of the 26.3 million articles indexed on PubMed
  were available on the site's repository of free materials, PubMed Central.
  Access to paywall-protected literature remains primarily through institutional
  subscriptions. Such subscriptions are costly and many struggle to afford
  access. The result is a disparity in access to the medical literature,
  particularly for those in low-income and middle-income countries (LMICs)%
}
According to \citeauthor{till:medical-literature} the high subscription cost to access medical
journal articles means that those in less advantaged countries cannot afford access to the
material, which is detrimental to the furthering of medical sciences. In light of this, it
should come as no surprise that \textcquote{till:medical-literature}[.]{
  Nearly 1 million articles published by medical journals are downloaded on
  Sci-Hub each month%
} Piracy enables the less fortunate countries to access valuable medical data that is
essential to furthering medical research for the human race. In such a specific case as
this, it should not be controversial to conclude that piracy is a good thing.

Medical journals are not the only source of information that can sit behind a paywall.
Sometimes, as in the previous case, there is information that one might argue is unjustly
kept away from the common man. What one considers to be unjust is subjective, but the idea
is that piracy is a tool that consumers can employ to fight systems that they deem to be
unjust. Without piracy, one might have to roll over and accept their limited access or
resort to a more extreme and likely illegal approach. To put it another way, piracy acts
as a check on businesses to discourage them from integrating unpopular, anti-consumer
practices into their business model. When a consumer wants to purchase something and they
cannot afford it, they might start to look for alternative way of obtaining it. But, when
they can afford it and they think it is fairly priced, they will purchase it. Thus, piracy
is often where tech-savvy persons turn to when they need something that they feel is
exorbitantly priced, but is not often utilized when they feel the purchase is worthwhile.
This way, piracy acts as a safeguard for consumers when they feel they are being taken
advantage of by predatory business practices.
% XXX the statements in the paragraph above are largely unsubstantiated >:D
% XXX add darnton:pirating-and-publishing and bohannon:everyone sources if needed

\subsection{Governance}
Given that piracy is an illegal usage of another person's conception, it has much to say
in the way of governance and intellectual property laws. The tools of piracy provide the
means of subverting an oppressive regime and call into question the legitimacy of
intellectual property laws.

Having a very firm grip on what citizens can see on the internet is one of the key
characteristics of a despotic government. Such governments know that if they can control
the flow of information on the Internet, it becomes much easier to keep the populace under
control. The government's imposing of restrictions on the Internet is clearly seen when
\textcquote{current:jigsaw}[.]{%
  Tunisia significantly ramped up its already aggressive blocking of specific
  websites in response to unrest that would ultimately unseat its government%
} Even more extreme is when governments shutdown the internet entirely to
quickly cease the spread of information. This is exactly what happened on
January 28, 2011 in Egypt: \textcquote{current:jigsaw}[.]{%
  Almost simultaneously, about 3,500 individual Border Gateway Protocol routes
  \textelp{} were withdrawn on orders from the Egyptian government, cutting the
  country off from the rest of the world and bringing internal communication to
  a halt%
} Fortunately, as already mentioned earlier, piracy excels at spreading
information even when there are those that do not want it spread. BitTorrent is
a peer-to-peer file sharing protocol that is often used when pirating content on the
Internet. A peer-to-peer protocol, by its very nature, does not require a centralized
server to store all the files that are to be shared. Instead, the source of the file is
each person who has the file downloaded. What this means for those under the rule of an
oppressive government is that they can spread information to others in such a way that the
government cannot easily block its source. Pieces of the information are sent to the one
downloading the file from everyone who had the file. There is no single source of the
information that the government can block to stop the spread. By a similar token, there
are peer-to-peer instant messaging applications that do not require an internet connection
to function; you need only be in the vicinity of other people that the message can hop
through to get to the destination. Both of these tools, in the context of piracy, facilitate
the illegal spread of information, but in the context of a despotic government, can be
employed to subvert their attempts at quelling the dispersal of information that could be
harmful to their rule.

Regarding piracy and intellectual property laws, it is clear that the two are at odds with
one another. That being said, it is worth examining whether IP laws accomplish what they
seek to do: encourage innovation. \citeauthor{lemley:faith-based} presents an argument
against IP laws when he writes, \blockcquote[1339]{lemley:faith-based}[.]{%
  \textins{IP} intervenes in the market to interfere with the freedom of others
  to do what they want in hopes of achieving the end of encouraging creativity.
  If we take that purpose out of the equation, we are left with a belief system
  that says the government should restrict your speech and freedom of action in
  favor of mine, not because doing so will improve the world, but simply because
  I spoke first%
}
To put it differently, intellectual property laws are based on the assumption that in
restricting who is allowed to have certain information, they are encouraging creativity
when, in reality, creativity does not always come first. IP laws are supposed to make
people want to work hard to get a good idea that they can have exclusive ownership of, but
having exclusive rights to an idea stifles further innovation. Under this system, only the
owner has the right to make changes. In fact, it seems that there would be more creativity
without IP laws. If everything was open to scrutiny and replication, there is bound to be
some improvement eventually because of all the people that could have a hand in making it
better. If a big streaming platform were to forfeit one of their IPs to the public, there
is no telling what kind of content could be created. In this way, piracy could allow for
the incremental improvement and scrutiny of conceptions, thus making them better.

\subsection{Privacy}
It would be remiss to discuss the tools of piracy without mentioning virtual private
networks (VPNs) and Tor, formerly known as the onion router. Commercially available VPNs
allow internet users to route their traffic through another machine on the internet before
connecting to the final destination host. What this means to the user is that their
traffic appears to be coming from another IP address than the one for their home network.
Since IP addresses are associated with geographical locations, a VPN can also mask a
user's location from bad actors. However, VPNs are not a perfect solution to privacy. The
company hosting the virtual private network is entrusted with all the user's internet
traffic, so the company should not be left susceptible to cybersecurity attacks or
coercion from government entities that might compromise someone's privacy online. In
general, the user should have confidence in the VPN provider that their information will
not be misused.

In the context of piracy, a VPN enables users to bypass firewalls that block access to
certain locations on the internet and they are employed in combination with the BitTorrent
protocol to protect one's identity when they are connecting to seeders or leechers. When
used responsibly, a privacy-conscious VPN provider is quite the boon to internet users in
countries that oppress certain people groups. As long as the VPN itself is not
compromised, a user's true identity can stay hidden. In the case of a missionary in a
hostile environment, a VPN might enable them to bypass the restrictions that prevent them
from accessing certain online editions of the Bible. Ultimately, VPNs have the ability to
be a tool for good in situations where access to information is limited and where one
might be persecuted if their identity were to be discovered on the internet.

In the same vein as a VPN is Tor. Tor is a service that allows users to browse the
internet without anyone but themselves knowing both the source and destination of a
connection. Unlike a VPN, there is no central company that can see who is connecting to
what destination; the only person who knows that is the one using the service. The Tor
Project writes, \textcquote{tor:about}[.]{%
  \textins{t}he goal of onion routing was to have a way to use the internet with
  as much privacy as possible, and the idea was to route traffic through
  multiple servers and encrypt it each step of the way%
} Tor not only provides privacy, it gives users total anonymity. Similar to
using a VPN, connecting to the Tor network before browsing the internet makes it
near impossible for observers to see what sites someone is accessing and for the
site to see where the request is coming from.  The potential applications for
near-perfect anonymity on the internet are innumerable. Those living under
oppressive governments can bypass censorship to reach the truth and Christians
can anonymously access the Bible online in restrictive countries, just to name a
few.

As a side note, there are instances where pirated software can be more privacy-oriented
than the paid version. Sometimes the pirated version is \emph{cracked} to remove certain
digital rights management (DRM) tools that often run background tasks unbeknownst to the
user. These background tasks often leak bits of information about the user in order to
validate that they are legitimate owners of the software. By collecting information about
users, a business invades the privacy of its user base. However, pirated and cracked
software does not have such DRM because it needed to be removed to allow it to be freely
distributed. Thus, when it comes to software that contains aggressive DRM, piracy offers
better security by default. Unfortunately, some pirated software introduces malicious
code, a subject that will be explored in the next section.

\section{Arguments Against Piracy}
While piracy enables free access to information and content, it is crucial to
faithfully and charitably explore the potential dangers and economic
implications associated with this practice. The arguments against piracy are
not easy, nor should be, easy to dismiss. This paper will focus mainly on the
ethical and economical concerns that the use of piracy introduces into society.

\subsection{Access to Dangerous/Illegal Content}
Piracy has been associated with unrestricted access to information and
content. As previously discussed, this aspect of piracy provides the means for
those in need to access information that would otherwise be unavailable to them;
similarly, piracy can thus provide avenues to bypass governmental restrictions
on public information. This same benefit, however, becomes a strong argument
against piracy when what is distributed is dangerous or illegal content.

\subsubsection{Criminality of the Seeder and Leecher}
This paper has focused on a more wholistic definition that encompasses
the tools, techniques, motivations, and content, but the motivations and
content that are involved in piracy should not be ignored. Piracy does involve
a form of theft and dissemination of an individual's or organization's
intellectual property. Piracy, thus can and most often does violate laws in
most jurisdictions. Piracy, when it uses torrenting, involves both the
seeder and the leecher (or the uploader and downloader, if not using torrenting).
Depending on the legislation of the country in question, the leecher may or
may not be found guilty of a crime, but this is complicated by the global
context of digital piracy.

In the United States, the downloader/streamer/leecher are generally considered
offenders that could be subject of fines or even criminal charges. The severity
of the penalties of piracy in these cases pose a major argument against the use
of piracy, at least in the United States or nations where the individual
downloading can be found guilty of a crime. \emph{Fair use} and statutes of
limitations may alleviate penalties, but repeat offences may also lose one's
favor in these cases.~\cite{felonies.org}

\subsubsection{Response}
An argument that focuses solely on the criminality of the actions of piracy
may present a strong deterrent for many that would consider piracy. It is true
that the repercussions of piracy can be great, and it is true that in many
cases both parties may be liable for damages or in danger of prison sentences.
Christians should consider these things deeply. We are called to be meek,
merciful, and peaceful (Matthew 5:5, 7, 9; Galatians 5:22-26). We are also
called to harmony (Romans 12:16--18, Galatians 5:15-17). Christians have
pondered on the questions of civil disobedience. While we are called to peace,
love, and obedience, our obedience to man should not disrupt our obedience to
God. Clearly, the Bible does not call for anarchy (see Romans 13), but it
does present some cases in which civil disobedience for a greater good (that
aligns with God's values) supersedes extreme patriotism.~\footnote{%
  See the examples of the Hebrew midwives of Exodus 1; see also Rahab's defence
  of the Israelite spies; see also Daniel's disobedience to King
  Nebuchadnezzar; see also examples of civil disobedience in Revelation 13:15%
} Christians have some precedent for civil disobedience, but, at the same time,
Christians should also expect that there will be punishments for our actions.

As the arguments against piracy that follow in this paper, one should consider
that legislation against piracy is not always evil; there are valid concerns
with piracy's immoral/illegal content and economic damages. Christians should
be careful to call upon civil disobedience as a defense for their piracy habits
online. On the other hand, we also need to consider some cases in which being
a seeder could benefit our neighbors with less resources to legally obtain
research, software, or files that could benefit the general population.

\subsubsection{Malware}
The intersection of trust and computing is not a new topic.
\citeauthor{thompson:trust}, the creator of the Unix operating system, made this
the topic of his Turing Award lecture. In his lecture, he poses the question,
\textcquote[761]{thompson:trust}[.]{%
  to what extent should one trust a statement that
  a program is free of Trojan Horses? Perhaps it is more important to trust the
  people who wrote the software%
} With piracy, however, we add an additional actor in this chain of trust.
\citeauthor{thompson:trust} focuses solely on the developer-to-user relationship
of trust, with the assumption of an open-source project.
\citeauthor{thompson:trust}, thus, comments on how
\textcquote[763]{thompson:trust}{%
  \textins{s}uch blatant code would not go undetected for long. Even the most
  casual perusal of the source of the C compiler would raise suspicions%
} When piracy is the method of distribution, however, the assumption of
equality between the source code and the delivered software does not hold. The
pirated software may have been tampered with, and the user has no simple way to
detect changes in the file(s) without doing some (often) complicated forensic
work. Although anti-malware tools may mitigate these issues, the problem of
trust still holds, and the anti-malware tools are not perfect. The character of
the seeder must be questioned, and it already stands on shaky ground if the
legality of their actions speaks for their moral compass and trustworthiness.

Piracy may openly distribute dangerous illegal content, but arguably the
larger threat is the dangerous content that remains undisclosed.
The relationship between piracy and malware distribution is not just a
theoretical argument against the use of piracy. Although research has remained
rather theoretical in this area, some researchers have performed empirical
studies to find a stronger link between more prevalent use of piracy and
malware presence. The study by \citeauthor{mezzour:empirical-study} reveals that
piracy is not equal around the globe, and it poses serious problems especially
for people in poor countries where piracy is widespread. Although they
hypothesized, like other researchers, that \enquote{cyber criminals would target
rich countries} because of monetary incentives, they found that, cyber criminals
targeted \textcquote{mezzour:empirical-study}[.]{%
  poor countries in Sub-Saharan Africa because of the low cost of attacking
  computers in this region%
} \citeauthor{mezzour:empirical-study} concludes that, due to the high
correlation between piracy and malware, one of the policy implications is that
Sub-Saharan Africa requires reducing piracy to combat the spread of malware.
In more recent news, Antivirus provider Kaspersky recently uncovered an example
of malware spreading through pirated software.~\cite{pcmag:malware}

\subsubsection{Response}
An argument based on malware distribution presents a strong case against piracy.
Moreover, we have to acknowledge that the problem targets those with less
resources (whether that be information, financial resources, or technical
resources) to protect themselves from the threats that malware poses. Piracy
circumvents mainline distribution channels that may incorporate secure policies
of data integrity, authenticity, and privacy. However, there are piracy tools
and sites that provide some ways to vet the authenticity of a user based on the
user's previous history and relationship with the piracy community. For example,
\emph{ThePirateBay}, one of the most popular pirating sites, displays the status
or class of their users and holds these users to higher standards in order to
retain trust with the site's end users.~\cite{suprbay:status}

Nevertheless, not all sites provide these metrics or warnings for the average
end-user. Instead, the burden is often on the end-user, and it is not unfair to
assume that the average person would not be able to analyze downloaded software
for malware. Thus, to safely make use of piracy tools, one must be aware of the
risks and the ways that they can mitigate those. This is not to say that users
need to reverse engineer all the software they download to look for malicious
behavior, but they do need to be aware of software sources that give them a
much better shot at getting an untainted copy. As mentioned previously, user
trust is a crucial component of piracy; those who choose to enter this domain
should be careful to download only from those with a high reputation in the
community---backed by a long time of faithful service.

The issue of trust is one that should not be unfamiliar to Christians. As
Christians, we have a Lord that we can trust in with all our hearts (Prov. 3:5).
As the Psalmist writes, \enquote{It is better to take refuge in the Lord than
to trust in man} (Ps. 118:8). Christians should be slow to put their entire
trust in metrics like those provided by \emph{ThePirateBay}; while they provide
some founts of trust, they should not immediately encourage Christians to trust
piracy.

\subsubsection{Illegal and Immoral Content}
Piracy, as discussed above, provides a way to distribute illegal content. This
content is not always beneficial to society; research articles, art, movies, \&c.
are not the only sorts of content that a pirate may distribute. Piracy opens
the door for questionable and illegal content that may be harmful to society
or national security. Since this paper aims to focus on a Christian exploration
of piracy, then this section will focus mostly on immoral content to build a
stronger argument against piracy that diversifies itself from an argument from
criminality (as was explored earlier in this paper).

There are few (if any) types of illegal content that are as immoral as that of
child pornography. It is a very sad reality, but this type of content is
distributed widely through the use of the Internet. Why and how this happens is
perhaps out of scope for this paper, but part of may be due to a sense of
privacy and anonymity that is fostered to an extreme in certain subcultures of
the Internet.~\cite[588]{prichard:subcultures} Research reveals that piracy and
piracy-adjacent tools provide the means to distribute this disgusting content
to unimaginable scales. \citeauthor{prichard:subcultures} explored this area of
research, with the following conclusions:
\blockcquote[593]{prichard:subcultures}{%
  Based on longitudinal data collected between August and November 2010, this
  study screened for the search terms that still appeared in the isoHunt Top
  300 list in November after being observed during August. In summary, the
  study found that 158/162 search terms related to infringing content, while
  3/162 search terms related to child pornography and 1/162 to bestiality%
}
These numbers may seem small to many. To properly interpret the above numbers,
consider that: 1) they do not represent the quantity of files that are shared
and 2) the researchers focused on isoHunt (which operated on the surface web)
and did not explore other means to find torrent files (i.e., means that utilize
the anonymity of the dark web). The numbers, rather, represent the consistent
interest within certain online subcultures in the immoral content. This
consistency presents a really compelling argument for Christians to question
piracy and the online culture that promotes and uses these methods.

\subsubsection{Response}
As mentioned above, Christians should not quickly dismiss this argument against
piracy. The reality of the matter is that we are sinful creatures full of
evil thoughts (Matt. 15:9); we are in dire need of a savior to redeem us, not
by any good works because even our best works are like filthy rags (Isa. 64:6).
This argument is compelling because of its severity. A naive response would
argue that harmful content is distributed via the surface web and without
piracy anyways. The argument, however, holds in that piracy enables the
widespread nature of this content, so what is a proper response to this
argument? There may be ways to engage with piracy communities to self-regulate
or to build better communities, as is suggested by
\citeauthor{prichard:subcultures}.~\cite[595--596]{prichard:subcultures}

This may be an unsatisfactory and complex answer to a large and complex issue.
Piracy (namely in the form of torrenting) does, however, rely on peers to
seed the content, and if changes can be made in these subcultures of the
internet. Because of the (mostly) decentralized nature of piracy, the best hope
is for the popular sites and individuals to self-regulate the content that is
promoted or accepted. For Christians, one option to keep in mind or consider
is to evangelize or set an example of what good online behavior looks like.
These questions of regulation and changes in the online culture, however,
reside outside the scope of this paper, but they, nevertheless, remain (and
will remain) increasingly relevant.

\subsection{Economic}
Piracy, as a form of theft, is also able to cause real harm to businesses. The
act of piracy can and does involve stealing and distributing content from
creators and organizations that invested time, money, and effort into the
content. In this sense, piracy is stealing from the individuals' revenue and
the government's tax revenue.~\cite{congress:pirating-the-american-dream}

\blockcquote{mrad:effects-of-ip}[.]{%
  According to Gould and Gruben (1996),
  \textins{Intellectual Property Rights (IPRs)} protection stimulates economic
  growth if it is accompanied by a policy of trade liberalization. By
  encouraging initiatives to innovate, IPRs protection may influence the
  economic growth of an open country. Park and Ginarte (1997) found that IPRs
  protection affects economic growth indirectly by stimulating the accumulation
  of factors of production such as physical capital and R\&D capital%
}
Piracy, as proponents of this argument would contend, discourages innovation in
the capitalist marketplace. Conversely, they argue that protecting intellectual
property leads to growth and innovation. Piracy, then, undermines the means for
economic growth that would benefit the general population.

\subsubsection{Response}
While this argument does hold in many cases, the economic reality is more
complex. The argument assumes that pirates would pay for the content if piracy
was not an option, but that assumption should be brought into question. As
argued earlier in this paper, piracy can be a means to control the market for
real change, and pirates maybe do not consider purchasing the content in the
first place.

On the other hand, it is not completely out of the question to imagine a
pirate coming back to some creator (whether that is an individual or a
business) to pay for the content when they are finally able to. As this paper
showed earlier, many pirates are those who do not have the means to pay for
some of the content that they pirate, but perhaps they would if/when they
could. This possibility of turnaround should be the focus of further research
in the culture of piracy, as the economic argument (for either side) largely
depends on whether pirates would pay for content if they could.

While economic growth can and does benefit the general population, a Christian
response to piracy should take into consideration some of the more complex
aspects of the global marketplace. It is true that the growth of a business can
benefit the employees generally, but the well-being of the consumer should also
be factored into the equation. In the case of commodity software such as
movies, music, or games, the problem that leads to widespread piracy in some
regions is an economic problem; the prices for software or media is set for and
targeted towards an affluent audience. If the companies do not set regional
prices that make their offerings reasonably available to a population, then it
should not come as a surprise that piracy will be widespread. As Christians and
computer scientists, we should be generous to the poor and set regional prices
that make sense for consumers if piracy is negatively affecting the people
(e.g., employees) that we are responsible for (Prov. 14:31; 19:17; Lk. 12:33-34).

\section{Christian Response}
A Christian response to piracy should honestly take into account both sides of
the debate. As shown above, piracy does have some merit to it. Christians have
to admit that piracy-promoting tools do enable equal access for those in need,
but, at the same time, we must recognize that the human nature is not perfect,
but is full of evil thoughts (Matthew 15:19). Our human corruption manifests
itself in many ways through our lives and actions, but, as explored in this
paper, it also presents itself in the software we share. Our deceitful hearts
(Jeremiah 17:9) make it hard to trust the software we download, but reckless
use of piracy ignores advice and is therefore the way of the fool where people
fall (Proverbs 12:15, 11:14).

On the one hand, a Christian use of piracy should be respectful of authority.
We are called to do this regardless of punishment, but as a matter of
conscience (Romans 13:4--5). We are to exercise our freedom respectfully and
not as a cover-up for evil (1 Peter 2:13--7). In other words, we should consider
both the consequences we may incur and the image we portray carefully before
making a conclusion about the practice of piracy. A proper consideration of
piracy should not ignore the serious ramifications and associations that come
with it.

On the other hand, piracy, when used properly, is able to be used for good
causes that serve God rather than men (Acts 5:29). Academic articles,
especially from medical journals, are crucial for the development of humanity
and the saving of lives. In these cases, when capitalist motivators impede these
greater goods, one should not turn a blind eye, but instead we should remember
the poor and those in need (Proverbs 19:17, 22:9, 15:11; Galatians 2:10). Cases
like these present valid concerns in which piracy appears to provide a viable
solution to the problems brought about by human greed.

\section{Conclusion}
In closing, piracy and its related tools are not as cut and dry of an issue as one might
initially be led to believe. On one hand, piracy facilitates the spread of illegal goods
and services. Pirated software from an untrusted source may also contain malware and the
act of piracy itself, from one perspective, takes business and money away from companies.
On the other hand, piracy can be a powerful force for good. It assists in the
dissemination of information that is dangerous to oppressive regimes and it ensures that
people of all financial backgrounds can access and contribute to valuable research for the
benefit of everyone. More generally, the goal of piracy is to keep information free and
open to everyone; in that regard, it promotes equality of access to human thoughts.

\subsection{Final Thoughts / Responses}
By far, one of the most common arguments raised against piracy, as highlighted previously,
is that piracy steals money from businesses because it demotivates people from paying for
their products and services. While this may be true in many cases, the reality is a little
more nuanced. Someone who chooses to pirate a product rather than purchase it may not have
bought the product in the first place if piracy was not an option. On top of that, if
someone really enjoys using the pirated version of a piece of software, they might choose
to pay for the legitimate version of it to support the developers and get new updates
faster. Even if the user refuses to pay for software, distributing the pirated version of
software helps popularize it. When something gets really popular, people want to try it
and that leads to more people who might decide to support the company who develops it by
purchasing the real version.

Regardless of the potential benefit of piracy, businesses still see it as a clear threat
to their bottom line and try their best to protect their intellectual property at the
expense of the consumer's open access to information.
\blockcquote[264]{halbert:agendas}[.]{%
  Of additional concern is the fact that by making
  \textins{Intellectual Property} theft a national security threat without being
  clear about what actually constitutes intellectual property, not only does the
  U.S. government create a new reason for a militarized Internet, but it also
  sets the stage for companies to assert that a range of other activities from
  file sharing to producing counterfeit DVDs threaten national security and
  require further state intervention%
}

With how advanced piracy technology has become and how it will continue to improve, it
will not be going away anytime soon. With that in mind, Christians should make an effort
to seek the ways in which the tools of piracy enable the spread of the Gospel and how it
can be used as a salt and a light to those in less fortunate places---fellow Christians
and unbelievers alike.

\clearpage
\printbibliography[title=References]
\end{refsection}

\clearpage
\nocite{*}
\printbibliography[title=Bibliography]
\end{document}

% vim: set ts=2 sw=2 et:
