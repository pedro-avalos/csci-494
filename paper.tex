\documentclass[onecolumn, 12pt]{article}

\usepackage[T1]{fontenc}
\usepackage[utf8]{inputenc}
\usepackage{lmodern}

\usepackage[letterpaper, margin=1in]{geometry}
\usepackage{setspace}

\usepackage[english]{babel}
\usepackage{csquotes}
\usepackage{footnote}
\usepackage{endnotes}
\usepackage[bottom, multiple]{footmisc}
\usepackage[level]{fmtcount}

\usepackage{float}
\usepackage{graphicx}
\usepackage{wrapfig}
\graphicspath{ {./graphics} }

\usepackage[style=ieee, backend=biber]{biblatex}
\addbibresource{sources.bib}

\usepackage[pdfusetitle, hidelinks]{hyperref}

\makesavenoteenv{figure}
\SetCiteCommand{\cite}

\title{Sailing the High Seas:\\ A Christian Exploration of Piracy}
\author{Pedro J. Avalos Jim\'enez, Niklas Anderson}
\date{Wheaton College\\\today}

\begin{document}
\maketitle

\section{Introduction}

\subsection{Background Information}

\begin{displayquote}
  \textins{T}hat ideas should freely spread from one to another over the globe,
  for the moral and mutual instruction of man, and improvement of his
  condition, seems to have been peculiarly and benevolently designed by nature,
  when she made them, like fire, expansible over all space, without lessening
  their density at any point, and like the air in which we breathe, move, and
  have our physical being, incapable of confinement or exclusive appropriation.
  Inventions then cannot, in nature, be a subject of property.
  Thomas Jefferson~\cite{barlow:wine}
\end{displayquote}
It is generally accepted that piracy is an illegal activity, but what is often overlooked
is the legitimate role that piracy plays in a variety of contexts. When it comes to
subverting oppressive regimes, for instance, the tools that piracy provides are
indispensable and for protecting the interests of consumers from predatory business
practices, the abilities of piracy are unmatched. 

Needless to say, with every unregulated system comes abuse and piracy is no exception.
On top of that, piracy, on an individual level, is extremely difficult to combat because
of the immaterial nature of the items that would need to be protected. Ideas flow freely
between people especially when the Internet is involved, so stopping this flow becomes an
exponentially more difficult problem the more people have the restricted data. The ease
with which ideas propagate lead people like Thomas Jefferson to claim that they are
``incapable of confinement or exclusive appropriation.'' Consequently, two parties form:
those that support piracy for its free and open spread of all information and those that
see piracy as a threat to the further advancement of technology and ideas.

Before proceeding further into the issue of piracy, it is important to have a grasp of
what piracy means insofar as how this paper interprets it. The dictionary defines piracy
as ``the unauthorized use of another's production, invention, or conception especially in
infringement of a copyright.'' While this covers the technical definition of piracy, this
paper elects to utilize a more wholistic definition. Piracy not only includes the
unauthorized use of someone's work, but it includes the tools and techniques that are
commonly used to accomplish this task such as torrent trackers, VPNs, and torrent clients.

As a Christian, how should one approach such a tool and how can it be utilized for the
betterment of God's kingdom? In the paper that follows, piracy will be analyzed through a
number of lenses, culminating in a Christian response to this issue. In doing so, the hope
is that the reader attains a fresh, wholistic understanding of the potential value of
piracy.

\subsection{Thesis/Argument}

Piracy should not be hastily thrown aside.
Piracy is an excellent tool for disseminating information, services, content,
\&c. that may not have been available any other way.

\section{Arguments For Piracy}
As mentioned above, even though piracy is illegal, there is still merit to the tools of
piracy and the values behind them. The merits of piracy can be divided into three realms:
equal access to information, governance, and privacy. 
\subsection{Equal Access}
At its core, piracy is primarily concerned with accessing information of all kinds,
regardless of whether someone is authorized to access that information or not. This is
what piracy is best at and, contrary to popular belief, it is not always a bad thing. 

Take the restricted access to academic journal articles, for instance. Many academic
journals require all readers to pay not an insignificant sum to access research articles
that could contain insight that is valuable to the human race as a whole. Because of the
high cost of entry, certain groups are automatically excluded from reading and building
upon the research that has already been done. John Barlow puts the situation eloquently in
\textit{Selling Wine Without Bottles: The Economy of the Mind on the Global Net} when he
says, \textcquote{barlow:wine}{I am not comfortable with a model which will restrict
inquiry to the wealthy.} In the case of academic journal articles, why shouldn't anyone be
able to better themselves by consuming quality research articles? All humans should be
allowed to continue to learn more about God's creation without being bound by their
financial status. Just as how someone can go to the library and checkout any book that
tickles their fancy, they should be freely able to learn from the ongoing research of our
world. Of course one should recognize that not all information is safe or helpful to be
freely available on the internet, but that is a topic for a later section.

A more concrete instance of where paywalls can be directly harmful to the human race is
with regards to medical research. Till et al. stresses how important access to medical
research is in their article \textit{Who is pirating medical literature? A bibliometric
review of 28 million Sci-Hub downloads}. \blockcquote{till:medical-literature} {Access to
the medical literature is essential for both the practice of evidence-based medicine and
meaningful contribution to medical sciences. Nonetheless, only 12\% of newly published
papers are freely accessible online, and, as of 2014, only 3 million of the 26.3 million
articles indexed on PubMed were available on the site's repository of free materials,
PubMed Central. Access to paywall-protected literature remains primarily through
institutional subscriptions. Such subscriptions are costly and many struggle to afford
access. The result is a disparity in access to the medical literature, particularly for
those in low-income and middle-income countries (LMICs).}
According to Till et al. the high subscription cost to access medical journal articles
means that those in less advantaged countries cannot afford access to the material, which
is detrimental to the furthering of medical sciences. In light of this, it should come as
no surprise that \textcquote{till:medical-literature} {Nearly 1 million articles published
by medical journals are downloaded on Sci-Hub each month.} Piracy enables the less
fortunate countries to access valuable medical data that is essential to furthering
medical research for the human race. In such a specific case as this, it should not be
controversial to conclude that piracy is a good thing.

Medical journals are not the only source of information that can sit behind a paywall.
Sometimes, as in the previous case, there is information that one might argue is unjustly
kept away from the common man. What one considers to be unjust is subjective, but the idea
is that piracy is a tool that consumers can employ to fight systems that they deem to be
unjust. Without piracy, one might have to roll over and accept their limited access or
resort to a more extreme and likely illegal approach. To put it another way, piracy acts
as a check on businesses to discourage them from integrating unpopular, anti-consumer
practices into their business model. When a consumer wants to purchase something and they
cannot afford it, they might start to look for alternative way of obtaining it. But, when
they can afford it and they think it is fairly priced, they will purchase it. Thus, piracy
is often where tech-savvy persons turn to when they need something that they feel is
exorbitantly priced, but is not often utilized when they feel the purchase is worthwhile.
This way, piracy acts as a safeguard for consumers when they feel they are being taken
advantage of by predatory business practices.
% XXX the statements in the paragraph above are largely unsubstantiated >:D
% XXX add darnton:pirating-and-publishing and bohannon:everyone sources if needed

\subsection{Governance}
Given that piracy is an illegal usage of another person's conception, it has much to say
in the way of governance and intellectual property laws. The tools of piracy provide the
means of subverting an oppressive regime and call into question the legitimacy of
intellectual property laws.

Having a very firm grip on what citizens can see on the internet is one of the key
characteristics of a despotic government. Such governments know that if they can control
the flow of information on the Internet, it becomes much easier to keep the populace under
control. The government's imposing of restrictions on the Internet is clearly seen when
\textcquote{current:jigsaw} {Tunisia significantly ramped up its already aggressive
blocking of specific websites in response to unrest that would ultimately unseat its
government.} Even more extreme is when governments shutdown the internet entirely to
quickly cease the spread of information. This is exactly what happened on January 28, 2011
in Egypt: \textcquote{current:jigsaw} {Almost simultaneously, about 3,500 individual
Border Gateway Protocol routes [...] were withdrawn on orders from the Egyptian
government, cutting the country off from the rest of the world and bringing internal
communication to a halt.} Fortunately, as already mentioned earlier, piracy excels at
spreading information even when there are those that do not want it spread. BitTorrent is
a peer-to-peer file sharing protocol that is often used when pirating content on the
Internet. A peer-to-peer protocol, by its very nature, does not require a centralized
server to store all the files that are to be shared. Instead, the source of the file is
each person who has the file downloaded. What this means for those under the rule of an
oppressive government is that they can spread information to others in such a way that the
government cannot easily block its source. Pieces of the information are sent to the one
downloading the file from everyone who had the file. There is no single source of the
information that the government can block to stop the spread. By a similar token, there
are peer-to-peer instant messaging applications that do not require an internet connection
to function; you need only be in the vicinity of other people that the message can hop
through to get to the destination. Both of these tools, in the context of piracy, facilitate
the illegal spread of information, but in the context of a despotic government, can be
employed to subvert their attempts at quelling the dispersal of information that could be
harmful to their rule.

Regarding piracy and intellectual property laws, it is clear that the two are at odds with
one another. That being said, it is worth examining whether IP laws accomplish what they
seek to do: encourage innovation. Mark Lemley presents an argument against IP laws when he
writes, \blockcquote[1339]{lemley:faith-based}{[IP] intervenes in the market to interfere
with the freedom of others to do what they want in hopes of achieving the end of
encouraging creativity. If we take that purpose out of the equation, we are left with a
belief system that says the government should restrict your speech and freedom of action
in favor of mine, not because doing so will improve the world, but simply because I spoke
first.} 
To put it differently, intellectual property laws are based on the assumption that in
restricting who is allowed to have certain information, they are encouraging creativity
when, in reality, creativity does not always come first. IP laws are supposed to make
people want to work hard to get a good idea that they can have exclusive ownership of, but
having exclusive rights to an idea stifles further innovation. Under this system, only the
owner has the right to make changes. In fact, it seems that there would be more creativity
without IP laws. If everything was open to scrutiny and replication, there is bound to be
some improvement eventually because of all the people that could have a hand in making it
better. If a big streaming platform were to forfeit one of their IPs to the public, there
is no telling what kind of content could be created. In this way, piracy could allow for
the incremental improvement and scrutiny of conceptions, thus making them better.

\subsection{Privacy}
As a side note, there are instances where pirated software can be more privacy-oriented
than the paid version. Sometimes the pirated version is ``cracked'' to remove certain
digital rights management (DRM) tools that often run background tasks unbeknownst to the
user. These background tasks often leak bits of information about the user in order to
validate that they are legitimate owners of the software. By collecting information about
users, a business invades the privacy of its user base. However, pirated and cracked
software does not have such DRM because it needed to be removed to allow it to be freely
distributed. Thus, when it comes to software that contains aggressive DRM, piracy offers
better security by default. Unfortunately, some pirated software introduces malicious
code, a subject that will be explored in the next section.

\section{Arguments Against Piracy}
While piracy enables free access to information and content, it is crucial to
faithfully and charitably explore the potential dangers and economic
implications associated with this practice. The arguments against piracy are
not easy, nor should be, easy to dismiss. This paper will focus mainly on the
ethical and economical concerns that the use of piracy introduces into society.

\subsection{Access to Dangerous/Illegal Content}
Piracy has been associated with unrestricted access to information and
content. As previously discussed, this aspect of piracy provides the means for
those in need to access information that would otherwise be unavailable to them;
similarly, piracy can thus provide avenues to bypass governmental restrictions
on public information. This same benefit, however, becomes a strong argument
against piracy when what is distributed is dangerous or illegal content.

\subsubsection{Criminality of the Seeder and Leecher}
This paper has focused on a more wholistic definition that encompasses
the tools, techniques, motivations, and content, but the motivations and
content that are involved in piracy should not be ignored. Piracy does involve
a form of theft and dissemination of an individual's or organization's
intellectual property. Piracy, thus can and most often does violate laws in
most jurisdictions. Piracy, when it uses torrenting, involves both the
seeder and the leecher (or the uploader and downloader, if not using torrenting).
Depending on the legislation of the country in question, the leecher may or
may not be found guilty of a crime, but this is complicated by the global
context of digital piracy.

In the United States, the downloader/streamer/leecher are generally considered
offenders that could be subject of fines or even criminal charges. The severity
of the penalties of piracy in these cases pose a major argument against the use
of piracy, at least in the United States or nations where the individual
downloading can be found guilty of a crime. ``Fair use'' and statutes of
limitations may alleviate penalties, but repeat offences may also lose one's
favor in these cases.~\cite{felonies.org}

\subsubsection{Malware}
Piracy may openly distribute dangerous illegal content, but arguably the
larger threat is the dangerous content that remains undisclosed. Piracy
circumvents mainline distribution channels that may incorporate secure policies
of data integrity, authenticity, and privacy. Some piracy tools provide
some ways to vet the authenticity of an uploader based on past uploads or based
on the user's relationship with the piracy/torrenting
site\footnote{%
  \href{https://pirates-forum.org/Thread-ThePirateBay-Want-Trusted-VIP-Status-Pink-Green-Skull-See-Here}{ThePirateBay},
  for example provides icons to display whether a user is \emph{Trusted},
  \emph{VIP}, or \emph{Moderator}%
}, but not all sites provide these metrics or warnings for the average end-user.
Instead, the burden is often on the end-user, and it is not unfair to assume
that the average person would not be able to analyze downloaded software for
malware.

% TODO: Add an example of malware distributed via piracy

\subsubsection{Illegal content}
Piracy, as discussed above, provides a way to distribute illegal content. This
content is not always beneficial to society; research articles, art, movies, \&c.
are not the only sorts of content that a pirate may distribute. Piracy opens
the door for questionable and illegal content that may be harmful to society
or national security.

% TODO: Talk about ties to illegal organizations that are involved?

\subsection{Economic}
Piracy, as a theft, also causes real harm to businesses. The act of piracy
does involve stealing and distributing content from creators and organizations
that invested time, money, and effort into the content. In this sense, piracy
is stealing from the individuals' revenue and the government's tax
revenue.~\cite{congress:pirating-the-american-dream}

\textcquote{mrad:effects-of-ip}[.]{%
  According to Gould and Gruben (1996), IPRs protection stimulates economic
  growth if it is accompanied by a policy of trade liberalization. By
  encouraging initiatives to innovate, IPRs protection may influence the
  economic growth of an open country. Park and Ginarte (1997) found that IPRs
  protection affects economic growth indirectly by stimulating the accumulation
  of factors of production such as physical capital and R\&D capital%
}
Piracy, therefore, discourages innovation in the capitalist marketplace.

% TODO
\section{Christian Response}
A Christian response to piracy should honestly take into account both sides of
the debate. As shown above, piracy does have some merit to it. Christians have
to admit that piracy-promoting tools do enable 

% 1 Peter 2:13-17
% Romans 13:4-5

\begin{enumerate}
  \item Drawing distinctions: Paywall to information is different from
    withholding information for sake of security

  \item Paywalls provide the means of living for artists, writers, \&c. Providing
    ways to circumvent these when unnecessary may hinder their livelihoods.

  \item Betterment of mankind as a whole, should not be behind a paywall

  \item Piracy provides an avenue for missionary outreach; piracy provides
    means of secret dissemination within an oppressive regime.

  \item Piracy provides the means for access to content/information without
    much accountability
\end{enumerate}

\section{Conclusion}
In closing, piracy and its related tools are not as cut and dry of an issue as one might
initially be led to believe. On one hand, piracy facilitates the spread of illegal goods
and services. Pirated software from an untrusted source may also contain malware and the
act of piracy itself, from one perspective, takes business and money away from companies.
On the other hand, piracy can be a powerful force for good. It assists in the
dissemination of information that is dangerous to oppressive regimes and it ensures that
people of all financial backgrounds can access and contribute to valuable research for the
benefit of everyone. More generally, the goal of piracy is to keep information free and
open to everyone; in that regard, it promotes equality of access to human thoughts. 

\subsection{Final Thoughts / Responses}

By far, one of the most common arguments raised against piracy, as highlighted previously,
is that piracy steals money from businesses because it demotivates people from paying for
their products and services. While this may be true in many cases, the reality is a little
more nuanced. Someone who chooses to pirate a product rather than purchase it may not have
bought the product in the first place if piracy was not an option. On top of that, if
someone really enjoys using the pirated version of a piece of software, they might choose
to pay for the legitimate version of it to support the developers and get new updates
faster. Even if the user refuses to pay for software, distributing the pirated version of
software helps popularize it. When something gets really popular, people want to try it
and that leads to more people who might decide to support the company who develops it by
purchasing the real version.

Regardless of the potential benefit of piracy, businesses still see it as a clear threat
to their bottom line and try their best to protect their intellectual property at the
expense of the consumer's open access to information. \blockcquote[264]{halbert:agendas}
{Of additional concern is the fact that by making \textins{Intellectual Property} theft a
national security threat without being clear about what actually constitutes intellectual
property, not only does the U.S. government create a new reason for a militarized
Internet, but it also sets the stage for companies to assert that a range of other
activities from file sharing to producing counterfeit DVDs threaten national security and
require further state intervention.}

With how advanced piracy technology has become and how it will continue to improve, it
will not be going away anytime soon. With that in mind, Christians should make an effort
to seek the ways in which the tools of piracy enable the spread of the Gospel and how it
can be used as a salt and a light to those in less fortunate places---fellow Christians
and unbelievers alike.

\clearpage
\nocite{*}
\printbibliography
\end{document}

% vim: set ts=2 sw=2 et:
