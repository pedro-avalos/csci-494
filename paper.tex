\documentclass[onecolumn, 12pt]{article}

\usepackage[T1]{fontenc}
\usepackage[utf8]{inputenc}
\usepackage{lmodern}

\usepackage[letterpaper, margin=1in]{geometry}
\usepackage{setspace}

\usepackage[english]{babel}
\usepackage{csquotes}
\usepackage{footnote}
\usepackage{endnotes}
\usepackage[bottom, multiple]{footmisc}
\usepackage[level]{fmtcount}

\usepackage{float}
\usepackage{graphicx}
\usepackage{wrapfig}
\graphicspath{ {./graphics} }

\usepackage[style=ieee, backend=biber]{biblatex}
\addbibresource{sources.bib}

\usepackage[pdfusetitle, hidelinks]{hyperref}

\makesavenoteenv{figure}
\SetCiteCommand{\cite}

\title{Sailing the High Seas:\\ A Christian Exploration of Piracy}
\author{Pedro J. Avalos Jim\'enez, Niklas Anderson}
\date{Wheaton College\\\today}

\begin{document}
\maketitle

\section{Introduction}

\subsection{Background Information}

\begin{displayquote}
  \textins That ideas should freely spread from one to another over the globe, for the
    moral and mutual instruction of man, and improvement of his condition, seems to have been
    peculiarly and benevolently designed by nature, when she made them, like fire, expansible
    over all space, without lessening their density at any point, and like the air in which we
    breathe, move, and have our physical being, incapable of confinement or exclusive
    appropriation. Inventions then cannot, in nature, be a subject of property.  Thomas
    Jefferson ~\cite{barlow:wine}
\end{displayquote}


\subsubsection{Intellectual Property}

\begin{enumerate}
  \item Piracy on an individual level: Very hard to combat, because you are
    trying to protect things that are immaterial ideas. Ideas flow freely,
    especially in the Internet. Some people might argue that all ideas should
    be free.

  \item While some people might argue that piracy should not be allowed because
    it stifles innovation.
\end{enumerate}

\subsection{Thesis/Argument}

Piracy should not be hastily thrown aside.
Piracy is an excellent tool for disseminating information, services, content,
\&c. that may not have been available any other way.

\section{Arguments For Piracy}

\subsection{Equal Access}

\begin{enumerate}
  \item Piracy allows equal access to information
  \item \textcquote{barlow:wine}{I am not comfortable with a model which will restrict inquiry to the wealthy.}

  \item Increased prices are an artificial barrier to access certain
    information. Piracy provides a circumvention to said artificial barrier.
    Piracy gives consumers a way to fight systems they see as
    unfair/censorship. Piracy is a tool of the consumer to control/balance the
    economy~\cite{darnton:pirating-and-publishing, bohannon:everyone}

  \item People pirate medical content.~\cite{till:medical-literature}
    Medical information is valuable to everyone. Having medical information is
    crucial to advancements in medicine, but sometimes the cost is too high.
    The paywalls, in this case, are directly harmful to ends that are
    beneficial to everyone.
\end{enumerate}

\subsection{Governance}

\begin{enumerate}
  \item Dissemination of information through means of piracy and torrenting
    gives access to those within areas where Internet shutdowns are commonplace.

  \item Piracy provides a way around an oppressive regime, which are on the
    rise.~\cite{current:jigsaw}

  \item IP laws arbitrarily restrict free speech / invention just because
    someone came up with it first.~\cite[1339]{lemley:faith-based} They are
    based on the assumption that, in doing so, they are encouraging creativity
    (to be first to invent/create), when, in reality, creativity doesn't always
    come first, but may rather be an improvement upon the established. The most
    creative solution is not always the first one. IP laws and restrictions
    block slight innovations upon the existing intellectual property.
\end{enumerate}

\subsection{Privacy}

\begin{enumerate}
  \item Piracy gives you access to cracked versions of software that provide
    more sensible privacy/freedom than what is accessible
    legally.~\cite{stallman:right-to-read}
\end{enumerate}

\section{Arguments Against Piracy}

\subsection{Access to Dangerous/Illegal Content}
\begin{enumerate}
  \item Piracy generally enables free/open access to information.

  \item This can be a bad thing for dangerous information that can threaten
    lives/livelihoods.

  \item Piracy would facilitate the spread of dangerous information concerning
    national security, with few tools to mitigate and almost none to stop it.

  \item Piracy provides the means to access content that is questionable/illegal.

  \item When you are pirating, you need to be extremely careful with what you
    are getting. Not all sources are trusted sources; software can be infected
    with malware.

  \item Information and content can be dispersed by bad actors.
\end{enumerate}

\subsection{Economic}
\begin{enumerate}
  \item Businesses are losing money because of acts of piracy.
  \item Piracy is theft, in this sense. It is taking away from employee
    benefits and potential tax revenue for the government.

  \item Piracy takes money from creators and organizations behind content
    creation.~\cite{congress:pirating-the-american-dream}

  \item \textcquote{mrad:effects-of-ip}{According to Gould and Gruben (1996), IPRs protection stimulates economic growth if it is
    accompanied by a policy of trade liberalization. By encouraging initiatives to innovate, IPRs
    protection may influence the economic growth of an open country. Park and Ginarte (1997) found
    that IPRs protection affects economic growth indirectly by stimulating the accumulation of
  factors of production such as physical capital and R\&D capital.}

    Piracy discourages innovation in the capitalist marketplace.
\end{enumerate}

\section{Christian Response}

\begin{enumerate}
  \item Drawing distinctions: Paywall to information is different from
    withholding information for sake of security

  \item Paywalls provide the means of living for artists, writers, \&c. Providing
    ways to circumvent these when unnecessary may hinder their livelihoods.

  \item Betterment of mankind as a whole, should not be behind a paywall

  \item Piracy provides an avenue for missionary outreach; piracy provides
    means of secret dissemination within an oppressive regime.

  \item Piracy provides the means for access to content/information without
    much accountability
\end{enumerate}

\section{Conclusion}

\subsection{Final Thoughts / Responses}

\begin{enumerate}
  \item In response to economic argument: Would the pirate have bought the
    product/service in the first place?
\end{enumerate}

\subsection{Applications}

\subsection{Further Work}

\clearpage
\nocite{*}
\printbibliography
\end{document}

% vim: set ts=2 sw=2 et:
